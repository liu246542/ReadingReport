% !Tex root = ../main.tex

\chapter{在隐私保护上的应用}\label{chp:application}

% 在这一章,我们主要调研了近年来发表在安全顶会上的论文,

这一章我们主要介绍布隆过滤器及其衍生数据结构在隐私保护相关协议中的应用。
其中涉及到的协议包括可搜索加密、隐私信息检索和隐私集合运算。
% 其中
% 本章涉及到的协议包括可搜索加密、隐私信息检索和隐私集合运算。
% 我们挑选了近年来
% 我们挑选了近年来发表在顶会上的论文,通过
% 可搜索加密、隐私信息检索和隐私集合计算三个密码学协议,并结合近年来发表的工作

\section{在对称可搜索加密方面的应用}

\subsection{背景介绍}

% 可搜索加密的定义,分类,系统模型
% 可搜索加密(searchable encryption)针对的外包加密文件上的搜索问题。
随着近年来数据规模的不断增大,越来越多的个人和企业选择将本地文件外包到云平台(如 iCloud、Amazon S3)进行存储。
% 数字化程度不断提高,文件
存储在云端的文件不仅为用户节省了本地存储所需要的成本,避免文件丢失的风险,还能让用户随时随地通过互联网对文件进行搜索和访问,极大地提高了便利性。
但是将文件直接存放在云服务器中也大大增加文件泄露的风险。
% 对于服务器内部来说,云服务提供商可以直接获取存储的文件;而
一方面云服务提供者可以直接获取文件,另一方面由于云服务器处在公开的网络环境中,很容易受到外部攻击者的攻击。
一旦文件遭到泄露,用户的隐私也受到威胁。
保护用户文件隐私的直接方式是将文件在本地进行加密,再将加密后的文件进行上传。
但是服务器无法在加密后的文件上执行搜索,用户需要搜索时只能把所有文件下载下来才能完成,这就丧失了将文件存储在云端的意义。
% 上传之前对文件加密,
% 但是这样一来服务器便无法对数据进行搜索,丧失了

为了解决文件隐私和可搜索之间的矛盾,对称可搜索加密(Searchable Symmetric Encryption,SSE)~\cite{song2000practical,curtmola2006searchable}的概念被提出。
对称可搜索加密通过为加密数据构造安全索引实现隐私保护的关键词搜索。
如图所示,对称可搜索加密中包含用户和服务器两个实体。
在上传阶段,用户不仅需要上传加密文件,还需要上传对应的安全索引。
在搜索阶段,用户根据需要搜索的关键词生成搜索令牌(tokens),服务器使用搜索令牌检索得到加密的文件标识并返回给用户。

% 从对称可搜索加密的搜索过程可以看出,
我们假设服务器是半诚实的(semi-honest),即服务器会诚实地执行协议,但它同时会对尝试分析输入输出信息来推断用户的隐私。
相比其他隐私保护的搜索方案,对称可搜索加密方案能在效率和安全之间取得更好的平衡。
% 在效率和安全之间
一方面,基于属性保留加密(Property-Preserving Encryption)的方案~\cite{bellare2007deterministica}可以直接保留密文中的相等关系,从而实现高效的搜索。
但服务器可以通过分析密文上的相等信息来执行频率统计攻击并还原明文信息~\cite{naveed2015inferencea}。
在另一方面,基于通用密码学工具(如同态加密、安全多方计算以及不经意随机访问机)虽然能够提供较强的安全性,但这些工具要么在计算上开销非常大,要么存在较大的通信开销,直接应用到加密搜索场景会面临效率问题~\cite{ren2023searchable}。
% 同态加密或多方安全计算虽然能够提供较强的安全性,但同态加密方案需要
而对称可搜索加密通过将搜索过程转移到安全索引上,在允许有限信息泄露的同时提供了高效的搜索。

对称可搜索加密允许泄露的信息也被称作模式信息(pattern information),这些信息包括:
\begin{itemize}
  \item 搜索模式(search pattern),即两次搜索是否包含相同的搜索关键词。
  \item 访问模式(access pattern),即每次搜索能匹配到哪些加密结果。
  \item 数量模式(volume pattern),即每次搜索返回的结果数量。
\end{itemize}
% 允许泄露这些信息是为了
对称可搜索加密协议的安全性是定义在给定的模式信息之上的,也就是说如果我们称一个对称可搜索加密协议是安全的,那么除了允许泄露的模式信息之外,它不会泄露其他的任何信息。
% 所以说对称可搜索加密方案是牺牲一定的安全性来换取搜索效率。
近些年有大量工作~\cite{cash2015leakageabuse,blackstone2020revisiting,ning2021leap,kamara2022sok}集中关注于如何利用这些模式信息来设计相应的攻击,这些攻击被统称为泄露滥用攻击(Leakage-Abuse Attacks, LAAs)。
而我们前面的介绍的过滤器及其衍生数据结构正好可以用来隐藏特定的模式信息,从而避免受到对应的攻击。
以下我们将给出两个具体例子,介绍这些数据结构是如何用到对称可搜索加密之中的。

% 索引形式,倒排索引

\subsection{隐藏中间结果模式的 SSE}

这一节我们介绍的是


HXT~\cite{lai2018result}
% 本节我们介绍的协议


\subsection{隐藏数量模式的 SSE}

\section{在隐私信息检索方面的应用}

\subsection{背景介绍}


\subsection{方案介绍}

\section{在隐私集合运算方面的应用}

\subsection{背景介绍}

% \subsection{方案介绍}
\subsection{隐私集合求交的协议}

\subsection{隐私集合求并的协议}

\section{总结}
